\documentclass[]{article}
\usepackage{lmodern}
\usepackage{amssymb,amsmath}
\usepackage{ifxetex,ifluatex}
\usepackage{fixltx2e} % provides \textsubscript
\ifnum 0\ifxetex 1\fi\ifluatex 1\fi=0 % if pdftex
  \usepackage[T1]{fontenc}
  \usepackage[utf8]{inputenc}
\else % if luatex or xelatex
  \ifxetex
    \usepackage{mathspec}
  \else
    \usepackage{fontspec}
  \fi
  \defaultfontfeatures{Ligatures=TeX,Scale=MatchLowercase}
\fi
% use upquote if available, for straight quotes in verbatim environments
\IfFileExists{upquote.sty}{\usepackage{upquote}}{}
% use microtype if available
\IfFileExists{microtype.sty}{%
\usepackage{microtype}
\UseMicrotypeSet[protrusion]{basicmath} % disable protrusion for tt fonts
}{}
\usepackage{hyperref}
\hypersetup{unicode=true,
            pdftitle={Homework 3},
            pdfborder={0 0 0},
            breaklinks=true}
\urlstyle{same}  % don't use monospace font for urls
\IfFileExists{parskip.sty}{%
\usepackage{parskip}
}{% else
\setlength{\parindent}{0pt}
\setlength{\parskip}{6pt plus 2pt minus 1pt}
}
\setlength{\emergencystretch}{3em}  % prevent overfull lines
\providecommand{\tightlist}{%
  \setlength{\itemsep}{0pt}\setlength{\parskip}{0pt}}
\setcounter{secnumdepth}{0}
% Redefines (sub)paragraphs to behave more like sections
\ifx\paragraph\undefined\else
\let\oldparagraph\paragraph
\renewcommand{\paragraph}[1]{\oldparagraph{#1}\mbox{}}
\fi
\ifx\subparagraph\undefined\else
\let\oldsubparagraph\subparagraph
\renewcommand{\subparagraph}[1]{\oldsubparagraph{#1}\mbox{}}
\fi

\title{Homework 3}

\input{preamble.tex}
\title{Homework}



\begin{document}
\maketitle

Please answer the following questions in complete sentences in a clearly
prepared manuscript and submit the solution by the due date on
Gradescope.

Remember that this is a graduate class. There may be elements of the
problem statements that require you to fill in appropriate assumptions.
You are also responsible for determining what evidence to include. An
answer alone is rarely sufficient, but neither is an overly verbose
description required. Use your judgement to focus your discussion on the
most interesting pieces. The answer to ``should I include `something' in
my solution?'' will almost always be: Yes, if you think it helps support
your answer.

\hypertarget{problem-0-homework-checklist}{%
\subsection{Problem 0: Homework
checklist}\label{problem-0-homework-checklist}}

\begin{itemize}
\item
  Please identify anyone, whether or not they are in the class, with
  whom you discussed your homework. This problem is worth 1 point, but
  on a multiplicative scale.
\item
  Make sure you have included your source-code and prepared your
  solution according to the most recent Piazza note on homework
  submissions.
\end{itemize}

\hypertarget{problem-1-convexity-and-least-squares}{%
\subsection{Problem 1: Convexity and least
squares}\label{problem-1-convexity-and-least-squares}}

\begin{enumerate}
\def\labelenumi{\arabic{enumi}.}
\item
  Show that \(f(x) = \| \vb - \mA \vx \|^2\) is a convex function. Feel
  free to use the result proved on the last homework.
\item
  Show that the null-space of a matrix is a convex set. (A convex set
  satisfies the condition that, for every pair of points in the set, any
  point on the line joining those points is also in the set.)
\end{enumerate}

\hypertarget{problem-2-ridge-regression}{%
\subsection{Problem 2: Ridge
Regression}\label{problem-2-ridge-regression}}

The Ridge Regression problem is a variant of least squares:
\[ \MIN{}{\| \vb - \mA \vx\|_2^2 + \lambda \| \vx\|_2^2. } \] This is
also known as Tikhonov regularization.

\begin{enumerate}
\def\labelenumi{\arabic{enumi}.}
\item
  Show that this problem always has a unique solution, for \emph{any
  \(\mA\)} if \(\lambda > 0\), using the theory discussed in class so
  far.

  \begin{quote}
  This property is one aspect of the reason that Ridge Regression and
  used. It is also a common regularization method that can help avoid
  overfitting in a regression problem.
  \end{quote}
\item
  Use the SVD of \(\mA\) to characterize the solution as a function of
  \(\lambda\).
\item
  What is the solution when \(\lambda \to \infty\)?\\
  What is the solution when \(\lambda \to 0\).
\item
  Give the solutions of the least squares problem using the sports teams
  data when \(\lambda = 0\) and \(\lambda = \infty\). (Here, you will
  not need the constraint that the sum of entries is 1.) Any notable
  differences from the ranking giving in class?
\item
  Suppose that you only want to regularize one component of the
  solution, say, \(x_1\), so that your optimization problem is
  \[ \MIN{}{\| \vb - \mA \vx\|_2^2 + \lambda x_1^2. } \] Show how to
  adapt your techniques in this problem to accomplish this goal.
\end{enumerate}

\hypertarget{problem-3-thinking-about-constraints}{%
\subsection{Problem 3: Thinking about
constraints}\label{problem-3-thinking-about-constraints}}

Find the minimizer of \(f(x,y) = x^2 + 2y^2\) where also \(y = 5x + 2\)
(that is, we have added one linear constraint) and justify, as concisely
and correctly as you can, that you have found a solution. Do the terms
local and global minimizer apply? If so, use one of them to describe
your solution. Furthermore, evaluate the gradient of \(f\) at your
solution. What is notable about the gradient at your solution in
comparison with the solution where \(x,y\) have no constraints?

\hypertarget{problem-4-alternate-formulations-of-least-squares}{%
\subsection{Problem 4: Alternate formulations of Least
Squares}\label{problem-4-alternate-formulations-of-least-squares}}

Consider the constrained least squares problem:
\[ \MINone{\vr,\vy}{ \| \vr \|_2 } { \vr = \vb - \mC\vy } \] where
\(\mC \in \RR^{m \times n}\), \(n \le m\) and rank \(n\).

\begin{enumerate}
\def\labelenumi{\arabic{enumi}.}
\item
  Convert this problem into the standard constrained least squares form.
\item
  Form the augmented system from the Lagrangian as we did in class.
\item
  Manipulate this problem to arrive at the normal equations for a
  least-squares problem: \(\mC^T \mC \vy = \mC^T \vb\).\\
  Discuss any advantages of the systems at intermediate steps.
\end{enumerate}

\end{document}
