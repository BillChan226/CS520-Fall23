\documentclass[]{article}
\usepackage{lmodern}
\usepackage{amssymb,amsmath}
\usepackage{ifxetex,ifluatex}
\usepackage{fixltx2e} % provides \textsubscript
\ifnum 0\ifxetex 1\fi\ifluatex 1\fi=0 % if pdftex
  \usepackage[T1]{fontenc}
  \usepackage[utf8]{inputenc}
\else % if luatex or xelatex
  \ifxetex
    \usepackage{mathspec}
  \else
    \usepackage{fontspec}
  \fi
  \defaultfontfeatures{Ligatures=TeX,Scale=MatchLowercase}
\fi
% use upquote if available, for straight quotes in verbatim environments
\IfFileExists{upquote.sty}{\usepackage{upquote}}{}
% use microtype if available
\IfFileExists{microtype.sty}{%
\usepackage{microtype}
\UseMicrotypeSet[protrusion]{basicmath} % disable protrusion for tt fonts
}{}
\usepackage{hyperref}
\hypersetup{unicode=true,
            pdftitle={Homework 1},
            pdfborder={0 0 0},
            breaklinks=true}
\urlstyle{same}  % don't use monospace font for urls
\IfFileExists{parskip.sty}{%
\usepackage{parskip}
}{% else
\setlength{\parindent}{0pt}
\setlength{\parskip}{6pt plus 2pt minus 1pt}
}
\setlength{\emergencystretch}{3em}  % prevent overfull lines
\providecommand{\tightlist}{%
  \setlength{\itemsep}{0pt}\setlength{\parskip}{0pt}}
\setcounter{secnumdepth}{0}
% Redefines (sub)paragraphs to behave more like sections
\ifx\paragraph\undefined\else
\let\oldparagraph\paragraph
\renewcommand{\paragraph}[1]{\oldparagraph{#1}\mbox{}}
\fi
\ifx\subparagraph\undefined\else
\let\oldsubparagraph\subparagraph
\renewcommand{\subparagraph}[1]{\oldsubparagraph{#1}\mbox{}}
\fi

\title{Homework 1}

\input{preamble.tex}
\title{Homework}



\begin{document}
\maketitle

Please answer the following questions in complete sentences in a clearly
prepared manuscript and submit the solution by the due date on
Gradescope.

Remember that this is a graduate class. There may be elements of the
problem statements that require you to fill in appropriate assumptions.
You are also responsible for determining what evidence to include. An
answer alone is rarely sufficient, but neither is an overly verbose
description required. Use your judgement to focus your discussion on the
most interesting pieces. The answer to ``should I include `something' in
my solution?'' will almost always be: Yes, if you think it helps support
your answer.

\hypertarget{problem-0-list-your-collaborators}{%
\subsection{Problem 0: List your
collaborators}\label{problem-0-list-your-collaborators}}

Please identify anyone, whether or not they are in the class, with whom
you discussed your homework. This problem is worth 1 point, but on a
multiplicative scale.

\hypertarget{problem-1-some-quick-simple-theory}{%
\subsection{Problem 1: Some quick, simple
theory}\label{problem-1-some-quick-simple-theory}}

\begin{enumerate}
\def\labelenumi{\arabic{enumi}.}
\item
  What is \(1/\sqrt(x)\) when \(x = 10^9\)? What is the limit of the
  sequence \(1/\sqrt(x)\) as \(x \to \infty\)? What type of convergence
  is this?
\item
  Show, using the definition, that the sequence \(1 + k^{-k}\) converges
  superlinearly to \(1\).
\item
  Suppose we have a sequence \(x_{k+1} = \sqrt{x_k}\). Show that this
  sequence converges for all positive inputs \(\vx_0\)\_. What is the
  the rate of convergence.
\item
  Let \(z = 9\). Consider the sequence
  \[ x_{k+1} = (1/2)  x_k (3 - z x_k^2), x_0 = 1/2. \] Show that this
  converges and give the convergence type (algebra, linear, quadratic,
  etc.)
\item
  (A little more interesting, note that I haven't yet worked out the
  answer to this one; it may be well-known in other areas, etc. Don't
  feel obliged to compute a full solution, but for full points you must
  show your investigation. ) Suppose we have a sequence
  \(x_{k+1} = |\log{x_k}|\). Does this converge for any input? For all
  inputs? For some? What else can you say about it? Are there limit
  points?
\end{enumerate}

\hypertarget{problem-2-angled-raptors}{%
\subsection{Problem 2: Angled raptors}\label{problem-2-angled-raptors}}

Mr.~Munroe (the xkcd author) decided that running in a single direction
was unrealistic. Your new problem is to solve the generalized raptor
problem where you can turn twice. Once at 0.25 seconds. A second time at
0.75 seconds. Otherwise the problem is the same.

\begin{itemize}
\tightlist
\item
  Ignore all acceleration, like we did in class.
\item
  The slow raptor runs at 10 m/s
\item
  The fast raptors run at 15 m/s
\item
  You run at 6 m/s
\item
  A raptor will catch you if you are within 20 centimeters.
\item
  You can turn instantaneously.
\end{itemize}

There are three parts.

\begin{enumerate}
\def\labelenumi{\arabic{enumi}.}
\item
  Modify the the Raptor chase example function to compute the survival
  time of a human in a raptor problem where you switch direction at 0.25
  seconds and 0.75 seconds. Show your modified function, and show the
  survival time when running directly at the slow raptor (up to time
  0.25) and then reversing your direction and running away from it.
\item
  Utilize a grid-search strategy to determine the best angles for the
  human to run to maximize the survival time. Show the angles.
\item
  Discuss the major challenge for solving this problem with the current
  strategy if we added a fourth angle at 0.5 seconds. (Or do so!) And a
  fifth angle at 0.75 seconds. (Or if you are feeling ambitious, solve
  these and see where you start running into trouble and discuss why
  that is!
\end{enumerate}

\end{document}
