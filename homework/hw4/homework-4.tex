\documentclass[]{article}
\usepackage{lmodern}
\usepackage{amssymb,amsmath}
\usepackage{ifxetex,ifluatex}
\usepackage{fixltx2e} % provides \textsubscript
\ifnum 0\ifxetex 1\fi\ifluatex 1\fi=0 % if pdftex
  \usepackage[T1]{fontenc}
  \usepackage[utf8]{inputenc}
\else % if luatex or xelatex
  \ifxetex
    \usepackage{mathspec}
  \else
    \usepackage{fontspec}
  \fi
  \defaultfontfeatures{Ligatures=TeX,Scale=MatchLowercase}
\fi
% use upquote if available, for straight quotes in verbatim environments
\IfFileExists{upquote.sty}{\usepackage{upquote}}{}
% use microtype if available
\IfFileExists{microtype.sty}{%
\usepackage{microtype}
\UseMicrotypeSet[protrusion]{basicmath} % disable protrusion for tt fonts
}{}
\usepackage{hyperref}
\hypersetup{unicode=true,
            pdftitle={Homework 4},
            pdfborder={0 0 0},
            breaklinks=true}
\urlstyle{same}  % don't use monospace font for urls
\IfFileExists{parskip.sty}{%
\usepackage{parskip}
}{% else
\setlength{\parindent}{0pt}
\setlength{\parskip}{6pt plus 2pt minus 1pt}
}
\setlength{\emergencystretch}{3em}  % prevent overfull lines
\providecommand{\tightlist}{%
  \setlength{\itemsep}{0pt}\setlength{\parskip}{0pt}}
\setcounter{secnumdepth}{0}
% Redefines (sub)paragraphs to behave more like sections
\ifx\paragraph\undefined\else
\let\oldparagraph\paragraph
\renewcommand{\paragraph}[1]{\oldparagraph{#1}\mbox{}}
\fi
\ifx\subparagraph\undefined\else
\let\oldsubparagraph\subparagraph
\renewcommand{\subparagraph}[1]{\oldsubparagraph{#1}\mbox{}}
\fi

\title{Homework 4}

\input{preamble.tex}
\title{Homework}



\begin{document}
\maketitle

Please answer the following questions in complete sentences in a clearly
prepared manuscript and submit the solution by the due date on
Gradescope.

Remember that this is a graduate class. There may be elements of the
problem statements that require you to fill in appropriate assumptions.
You are also responsible for determining what evidence to include. An
answer alone is rarely sufficient, but neither is an overly verbose
description required. Use your judgement to focus your discussion on the
most interesting pieces. The answer to ``should I include `something' in
my solution?'' will almost always be: Yes, if you think it helps support
your answer.

\hypertarget{problem-0-homework-checklist}{%
\subsection{Problem 0: Homework
checklist}\label{problem-0-homework-checklist}}

\begin{itemize}
\item
  Please identify anyone, whether or not they are in the class, with
  whom you discussed your homework. This problem is worth 1 point, but
  on a multiplicative scale.
\item
  Make sure you have included your source-code and prepared your
  solution according to the most recent Piazza note on homework
  submissions.
\end{itemize}

\hypertarget{problem-1-log-barrier-terms}{%
\subsection{Problem 1: Log-barrier
terms}\label{problem-1-log-barrier-terms}}

The basis of a class of methods known as interior point methods is that
we can handle non-negativity constraints such as \(\vx \ge 0\) by
solving a sequence of unconstrained problems where we add the function
\(b(\vx; \mu) = -\mu \sum_i \log(x_i)\) to the objective. Thus, we
convert \[ \MINone{}{f(\vx)}{\vx \ge 0} \] into
\[ \MIN{}{f(\vx) + b(\vx; \mu).} \]

\begin{enumerate}
\def\labelenumi{\arabic{enumi}.}
\item
  Explain why this idea could work. (Hint: there's a very useful picture
  you should probably show here!)
\item
  Write a matrix expression for the gradient and Hessian of
  \(f(\vx) + b(\vx; \mu)\) in terms of the gradient vector \(g(\vx)\)
  and the Hessian matrix \(\mH(\vx)\) of \(f\).
\item
  (Open ended) Explain if there is anything special about \(\log\) that
  makes it especially useful here. For instance, are there other
  functions that could work and give you the same effect? Would \(\log\)
  be superior?
\end{enumerate}

\hypertarget{problem-2-inequality-constraints}{%
\subsection{Problem 2: Inequality
constraints}\label{problem-2-inequality-constraints}}

Draw a picture of the feasible region for the constraints:
\[ \bmat{1 - x_1 - x_2 \\ 1 - x_1 + x_2 \\ 1 + x_1 - x_2 \\ 1 + x_1 + x_2} \ge 0. \]

\hypertarget{problem-3-necessary-and-sufficient-conditions}{%
\subsection{Problem 3: Necessary and sufficient
conditions}\label{problem-3-necessary-and-sufficient-conditions}}

Let \(f(\vx) = \frac{1}{2} \vx^T \mQ \vx - \vx^T \vc\).

\begin{enumerate}
\def\labelenumi{\arabic{enumi}.}
\item
  Write down the necessary conditions for the problem:
  \[ \MINone{}{f(\vx)}{\vx \ge 0}. \]
\item
  Write down the sufficient conditions for the same problem.
\item
  Consider the two-dimensional case with
  \[ \mQ = \bmat{2 & 3 \\ 3 & 2} \qquad \vc = \bmat{-1.5 \\ 1.5}. \]
  Determine the solution to this problem by any means you can, and
  justify your work.
\item
  Produce a Julia or hand illustration of the solution showing the
  function contours, and gradient. What are the active constraints at
  the solution? What is the value of \(\lambda\) in
  \(\mA^T \lambda = \vg\)?
\item
  What changes when we set \(\mQ = \bmat{-2 & 3 \\ 3 & -2}\)?
\end{enumerate}

\hypertarget{problem-4-constraints-can-make-a-non-smooth-problem-smooth.}{%
\subsection{Problem 4: Constraints can make a non-smooth problem
smooth.}\label{problem-4-constraints-can-make-a-non-smooth-problem-smooth.}}

Show that

\[ \MIN{}{\sum_i \max (\va_i^T \vx - b_i, -0.5)} \]

can be reformulated as a constrained optimization problem with a
continuously differentiable objective function and both linear equality
and inequality constraints.

\end{document}
